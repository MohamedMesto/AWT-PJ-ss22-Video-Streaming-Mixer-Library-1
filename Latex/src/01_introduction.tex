\section{\textbf{Introduction}}\label{sec:Introduction}
We are definitely in the streaming area. Live shows, movies, TV programs, music, and any other multimedia content are being delivered and consumed continuously from a source over the internet. The user can start playing the content without having to wait for it to be downloaded or transmitted. Adaptive bitrate streaming standards, like HLS, are being used nowadays to enhance the user experience while consuming video streams over the internet, changing and adapting dynamically the resolution of the video source to avoid buffering and bandwidth bottlenecks.

HLS provides different versions or copies of a video stream with lower, medium, and higher quality resolution. It splits a video source into little segments for each available resolution. Each resolution is a playlist itself containing all those split videos of their respective resolution. A user, for example, might want to enjoy multiple streams as one single playlist to avoid pausing each TV episode.

To combine these streams into the main one, they have to be compatible with each other in terms of video resolution. A video stream can have multiple resolutions available thanks to HLS, but not all streams share the same representations. Two video streams might not be compatible with each other because they may have different source resolutions from the HLS initial formatting.

We seek to resolve this issue by creating a playlist consisting of multiple video streams that already exist, making sure that the different video stream sources are all compatible in regards to resolution. We need to identify the multiple representations available from each video stream that is intended to be in the playlist. Then, having this data, an algorithm is needed to handle and filter out the video streams that are not compatible to output a single playlist with the final matching representation that includes multiple video stream sources.
